\documentclass[red, ccm]{slideccm}

% TÍTULO
\title{Título}
\subtitle{Subtítulo}
\author{Nome Completo}
\institute{Instituto}
\date{\today}

% COMANDOS PERSONALIZADOS
% \addbibresource{refs.bib}

% INÍCIO DO DOCUMENTO
\begin{document}

\frame{\titlepage}

\section{Texto}

\begin{frame}{Lista}
    \begin{itemize}
        \item Nunc non eleifend sem. Nunc consequat metus vel imperdiet vestibulum. In id ligula arcu. Orci varius natoque penatibus et magnis dis parturient montes, nascetur.
        \item Fusce nec ante nec lacus dapibus tincidunt sit amet quis arcu. Mauris at ultricies odio, id gravida mauris. Suspendisse.
        \begin{enumerate}
            \item Etiam eros erat, sagittis id mi id, sollicitudin.
            \item Ut et varius urna. Ut non leo iaculis, scelerisque sem eu.
        \end{enumerate}
    \end{itemize}
\end{frame}

\section{Matemática}

\begin{gather*}
    \NN = \{ 1, 2, 3, \dots \} \\
    \mathscr{X} = \{ x_1, x_2, \dots, x_N \} \\
    \int\limits_a^b {\frac{d}{dx} F(x) \, dx}  = F(b) - F(a) \\
    \int u \frac{dv}{dx} \, dx = uv - \int \frac{du}{dx} v \, dx \\
    \Gamma(x) = \int\limits_0^\infty s^{x - 1} e^{-s} \, ds \\
    -\frac{\hbar^2}{2m}\frac{d^2 \psi(x)}{dx^2} + U(x)\psi (x) = E \psi(x)
\end{gather*}

% \section{Código}

% \begin{minted}{python}
% # Python program for implementation of Quicksort Sort

% # This implementation utilizes pivot as the last element in the nums list
% # It has a pointer to keep track of the elements smaller than the pivot
% # At the very end of partition() function, the pointer is swapped with the pivot
% # to come up with a "sorted" nums relative to the pivot


% # Function to find the partition position
% def partition(array, low, high):

%     # choose the rightmost element as pivot
%     pivot = array[high]

%     # pointer for greater element
%     i = low - 1

%     # traverse through all elements
%     # compare each element with pivot
%     for j in range(low, high):
%         if array[j] <= pivot:

%             # If element smaller than pivot is found
%             # swap it with the greater element pointed by i
%             i = i + 1

%             # Swapping element at i with element at j
%             (array[i], array[j]) = (array[j], array[i])

%     # Swap the pivot element with the greater element specified by i
%     (array[i + 1], array[high]) = (array[high], array[i + 1])

%     # Return the position from where partition is done
%     return i + 1

% # function to perform quicksort


% def quickSort(array, low, high):
%     if low < high:

%         # Find pivot element such that
%         # element smaller than pivot are on the left
%         # element greater than pivot are on the right
%         pi = partition(array, low, high)

%         # Recursive call on the left of pivot
%         quickSort(array, low, pi - 1)

%         # Recursive call on the right of pivot
%         quickSort(array, pi + 1, high)


% data = [1, 7, 4, 1, 10, 9, -2]
% print("Unsorted Array")
% print(data)

% size = len(data)

% quickSort(data, 0, size - 1)

% print('Sorted Array in Ascending Order:')
% print(data)
% \end{minted}

\end{document}
